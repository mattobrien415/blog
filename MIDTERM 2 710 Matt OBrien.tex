%%%%%%%%%%%%%%%%%%%%%%%%%%%%%%%%%%%%%%%%%
% Short Sectioned Assignment
% LaTeX Template
% Version 1.0 (5/5/12)
%
% This template has been downloaded from:
% http://www.LaTeXTemplates.com
%
% Original author:
% Frits Wenneker (http://www.howtotex.com)
%
% License:
% CC BY-NC-SA 3.0 (http://creativecommons.org/licenses/by-nc-sa/3.0/)
%
%%%%%%%%%%%%%%%%%%%%%%%%%%%%%%%%%%%%%%%%%

%----------------------------------------------------------------------------------------
%	PACKAGES AND OTHER DOCUMENT CONFIGURATIONS
%----------------------------------------------------------------------------------------

\documentclass[paper=a4, fontsize=11pt]{scrartcl} % A4 paper and 11pt font size

\usepackage[T1]{fontenc} % Use 8-bit encoding that has 256 glyphs
\usepackage{fourier} % Use the Adobe Utopia font for the document - comment this line to return to the LaTeX default
\usepackage[english]{babel} % English language/hyphenation
\usepackage{amsmath,amsfonts,amsthm} % Math packages
\usepackage{verbatim} %for commenting out blocks

\usepackage{lipsum} % Used for inserting dummy 'Lorem ipsum' text into the template

\usepackage{sectsty} % Allows customizing section commands
\allsectionsfont{\centering \normalfont\scshape} % Make all sections centered, the default font and small caps

\usepackage{fancyhdr} % Custom headers and footers
\pagestyle{fancyplain} % Makes all pages in the document conform to the custom headers and footers
\fancyhead{} % No page header - if you want one, create it in the same way as the footers below
\fancyfoot[L]{} % Empty left footer
\fancyfoot[C]{} % Empty center footer
\fancyfoot[R]{\thepage} % Page numbering for right footer
\renewcommand{\headrulewidth}{0pt} % Remove header underlines
\renewcommand{\footrulewidth}{0pt} % Remove footer underlines
\setlength{\headheight}{13.6pt} % Customize the height of the header
%\numberwithin{equation}{section} % Number equations within sections (i.e. 1.1, 1.2, 2.1, 2.2 instead of 1, 2, 3, 4)
%\numberwithin{figure}{section} % Number figures within sections (i.e. 1.1, 1.2, 2.1, 2.2 instead of 1, 2, 3, 4)
%\numberwithin{table}{section} % Number tables within sections (i.e. 1.1, 1.2, 2.1, 2.2 instead of 1, 2, 3, 4)

\setlength\parindent{0pt} % Removes all indentation from paragraphs - comment this line for an assignment with lots of text
%----------------------------------------------------------------------------------------
%	TITLE SECTION
%----------------------------------------------------------------------------------------
\newcommand{\horrule}[1]{\rule{\linewidth}{#1}} % Create horizontal rule command with 1 argument of height
\title{	
\normalfont \normalsize 
%----------------------------------------------------------------------------------------
%	PUT YOUR NAME HERE 
%----------------------------------------------------------------------------------------
\textsc{Matt O'Brien} \\ [25pt] % Your university, school and/or department name(s)
\horrule{0.5pt} \\[0.4cm] % Thin top horizontal rule
%\huge  \\ % The assignment title
%\horrule{2pt} \\[0.5cm] % Thick bottom horizontal rule
}
%----------------------------------------------------------------------------------------
%	WHAT IS THE ASSIGNMENT NUMBER
%----------------------------------------------------------------------------------------
\author{MIDTERM \#2  \text{          }MATH 710} 
\date{\normalsize\today} % Today's date or a custom date
\begin{document}
\maketitle % Print the title
%----------------------------------------------------------------------------------------
%	CHEETSHEET
%----------------------------------------------------------------------------------------
\begin{comment}
\begin{align*}
\int_E (f \cdot \chi A) &= \int_{E/A} (f \cdot \chi A) + \int_A (f \cdot \chi A) \text{ ,by linearity properties of the integral}.
\\  &= 0  + \int_A (f \cdot \chi A) \text{ , by definition of characteristic function}
\\  &=  \int_A (f \cdot \ 1) \text{ , by definition of characteristic function}
\end{align*} 
ok here is the cheetshee for piecewise functions
$$
f(n) =
\begin{cases}
n/2, & \text{if }n\text{ is even} \\
3n+1, & \text{if }n\text{ is odd}
\end{cases}
$$
\end{comment}	

%\begin{comment}
%----------------------------------------------------------------------------------------
%	PROBLEM 1
%----------------------------------------------------------------------------------------
\section*{Exercise 1}
\boldmath
\textbf{Show that the function $f(x) = \frac{1}{\sqrt[3]{(1-x)}}$ is Lebesgue integrable on the interval $(0, 1)$ and evaluate the integral $\int_{0}^{1} \frac{1}{\sqrt[3]{(1-x)}}$ }
\unboldmath
\begin{proof}
First, note that the function is not Riemann integrable since the function is not bounded over $E = (0, 1)$.  To demonstrate Lebesgue integrability, first we let E be the union of a countable family of pairwise disjoint measurable sets.  Next, we define an increasing sequence of functions, bounded below $f$, where
$$
f_n = 
\begin{cases}
n, & \text{if } f > 1 - n^3 \\
f, & \text{if } f \leq 1 -n^3
\end{cases}$$
These functions are non negative and measurable over $E$.  Let $f_n$ converge to $f$ and note that $f_n$ must converge pointwise almost everywhere to $f$ since $f$ is unbounded at a single point.
\newline
We will show $f$ is Lebesgue integrable directly.
\newline
First observe that $f_n \leq f$ over $E$, for all $f_n$.
\newline
Thus, $\int_E f_n \leq \int_E f.$  Since $f$ is not bounded, we consider the lim sup of $f_n$.  We know that lim sup$\int_E f_n \leq \int_E f$.  Recall that the convergence is pointwise almost everywhere.  By Fatou's Lemma, $\int_E f \leq $lim sup$\int_E f_n \Rightarrow $lim inf$\int_E f_n = $lim sup$ \int_E f_n$.  Therefore, lim $\int_E f_n = \int_E $lim$ f_n = \int_E f$, which shows that the integral exists.
\newline
Notice that lim sup$\int_E f_n$ equals $\frac{3}{2}$, which equals $\int_E f$ and we are done.
\end{proof}

\begin{comment}
Consider the sequence of functions $(f_n)$, where each $f_n=f\cdot \chi_{(0, 1-\frac{1}{n})}$, where $\chi_{A}$ is the characteristic function of a set $A$ (p. 105). Observe that this is an increasing sequence of nonnegative measurable functions on $(0, 1)$ that converge pointwise to $f$ a.e. on $(0, 1)$. Furthermore, observe that each $f_n$ is Riemann integrable and is thus Lebesgue integrable with the same integral.

Then by Theorem 3.27 (p. 94), \[\int_0^1 f=\lim_{n\to\infty} \int_0^1 f_n=\lim \int_0^1 f\cdot \chi_{(0, 1-\frac{1}{n})}=\lim_{n\to\infty} (\int_0^{1-\frac{1}{n}} f+\int_{1-\frac{1}{n}}^1) f)=\lim_{n\to\infty} \int_0^{1-\frac{1}{n}} f=\]

\[\lim_{n\to\infty} (R) \int_0^{1-\frac{1}{n}} \frac{1}{\sqrt[3]{1-x}} \, dx.\]

Let $u=1-x$. Then $du=-dx$, or $-du=dx$. Observe that when $x=0, u=1-0=1$, and when $x=1-\frac{1}{n}, u=1-(1-\frac{1}{n})=\frac{1}{n}$.

Then \[\int_0^1 f=\lim_{n\to\infty} (R) \int_{\frac{1}{n}}^1 \frac{1}{\sqrt[3]{u}} \, du=\lim_{n\to\infty} [\frac{3}{2}u^{\frac{2}{3}}]_{\frac{1}{n}}^1=\lim_{n\to\infty} (\frac{3}{2}-\frac{3}{2}(\frac{1}{n})^{\frac{2}{3}})=\frac{3}{2}.\]
done.
\end{comment}

%----------------------------------------------------------------------------------------
%	PROBLEM 2
%----------------------------------------------------------------------------------------
\section*{Exercise 2}
\boldmath
\textbf{Let $E = \cup_{k=0}^{\infty}E_k$, where $E_k = (2^{-k-1}, 2^{-k}]$, and let $f(x) = (-1)^k$ for $x \in E_k$.  Evaluate the integral $\int f$ or show that the function $f$ is not integrable.}
\unboldmath
\begin{proof}
Observe that for $E_k = (2^{-k-1}, 2^{-k}]$, each subinterval is pairwise disjoint.  By the additive property of the integral (Theorem $3.7$), and Exercise 3.1a,
$$\int_E f = \sum_{k = 0}^{\infty} \int_{E_k} f = \sum_{k=0}^{\infty} c \cdot m(E_K) = \sum_{k=0}^{\infty} (-1)^k m(E_k).$$
\newline
Further,  
\begin{align*}
\sum_{k=0}^{\infty} (-1)^k m(E_k) &= (-1)^0 m(E_0) + (-1)^1 m(E_1) + (-1)^2 m(E_2) ...
\\ &=  m(E_0) - m(E_1) + m(E_2) + \dots
\end{align*}
Now, observe that $E_0 = \left( \frac{1}{2}, 1 \right]$, $E_1 = \left( \frac{1}{4}, \frac{1}{2} \right]$, $E_2 = \left( \frac{1}{8},\frac{1}{4} \right]$.
\newline
Thus, $$\sum_{k=0}^{\infty} (-1)^k m(E_k)  =  \frac{1}{2} - \frac{1}{4} + \frac{1}{8} + \dots$$
This is a geometric series whose first term is $\frac{1}{2}$ and whose common ration is $- \frac{1}{2}$, so it's sum is $\frac{1}{3}$ ; further, this is an alternating series and the convergence is absolute.
\end{proof}
%----------------------------------------------------------------------------------------
%	PROBLEM 3
%----------------------------------------------------------------------------------------
\section*{Exercise 3}
\boldmath
\textbf{Let $E = \cup_{k=1}^{\infty}$ where all sets $E_k$'s are measurable and $mE < \infty$.  Suppose that $f$ is bounded and integrable on every $E_k$.
\newline
True or False:  $f$ is integrable on $E$.  Justify your answer.}
\unboldmath
\begin{proof}
False:  to show that $f$ is not integrable on $E$, we furnish a counterexample:  
\newline
Suppose that $E = \cup_{k=1}^{\infty}$ , where $E_k = \left ( \frac{1}{k+1}, \frac{1}{k} \right ]$.  
\newline Consider $f = \frac{1}{x}$.  Notice that $f$ is continuous and bounded on each interval $E_n$.  But since $f$ is not integrable at $x=0$, $f$ is not integrable over $E$.
\end{proof}
\end{document}