%%%%%%%%%%%%%%%%%%%%%%%%%%%%%%%%%%%%%%%%%
% Short Sectioned Assignment
% LaTeX Template
% Version 1.0 (5/5/12)
%
% This template has been downloaded from:
% http://www.LaTeXTemplates.com
%
% Original author:
% Frits Wenneker (http://www.howtotex.com)
%
% License:
% CC BY-NC-SA 3.0 (http://creativecommons.org/licenses/by-nc-sa/3.0/)
%
%%%%%%%%%%%%%%%%%%%%%%%%%%%%%%%%%%%%%%%%%

%----------------------------------------------------------------------------------------
%	PACKAGES AND OTHER DOCUMENT CONFIGURATIONS
%----------------------------------------------------------------------------------------

\documentclass[paper=a4, fontsize=11pt]{scrartcl} % A4 paper and 11pt font size

\usepackage[T1]{fontenc} % Use 8-bit encoding that has 256 glyphs
\usepackage{fourier} % Use the Adobe Utopia font for the document - comment this line to return to the LaTeX default
\usepackage[english]{babel} % English language/hyphenation
\usepackage{amsmath,amsfonts,amsthm} % Math packages
\usepackage{verbatim} %for commenting out blocks

\usepackage{lipsum} % Used for inserting dummy 'Lorem ipsum' text into the template

\usepackage{sectsty} % Allows customizing section commands
\allsectionsfont{\centering \normalfont\scshape} % Make all sections centered, the default font and small caps

\usepackage{fancyhdr} % Custom headers and footers
\pagestyle{fancyplain} % Makes all pages in the document conform to the custom headers and footers
\fancyhead{} % No page header - if you want one, create it in the same way as the footers below
\fancyfoot[L]{} % Empty left footer
\fancyfoot[C]{} % Empty center footer
\fancyfoot[R]{\thepage} % Page numbering for right footer
\renewcommand{\headrulewidth}{0pt} % Remove header underlines
\renewcommand{\footrulewidth}{0pt} % Remove footer underlines
\setlength{\headheight}{13.6pt} % Customize the height of the header
%\numberwithin{equation}{section} % Number equations within sections (i.e. 1.1, 1.2, 2.1, 2.2 instead of 1, 2, 3, 4)
%\numberwithin{figure}{section} % Number figures within sections (i.e. 1.1, 1.2, 2.1, 2.2 instead of 1, 2, 3, 4)
%\numberwithin{table}{section} % Number tables within sections (i.e. 1.1, 1.2, 2.1, 2.2 instead of 1, 2, 3, 4)

\setlength\parindent{0pt} % Removes all indentation from paragraphs - comment this line for an assignment with lots of text
%----------------------------------------------------------------------------------------
%	TITLE SECTION
%----------------------------------------------------------------------------------------
\newcommand{\horrule}[1]{\rule{\linewidth}{#1}} % Create horizontal rule command with 1 argument of height
\title{	
\normalfont \normalsize 
%----------------------------------------------------------------------------------------
%	PUT YOUR NAME HERE 
%----------------------------------------------------------------------------------------
\textsc{Matt O'Brien} \\ [25pt] % Your university, school and/or department name(s)
\horrule{0.5pt} \\[0.4cm] % Thin top horizontal rule
%\huge  \\ % The assignment title
%\horrule{2pt} \\[0.5cm] % Thick bottom horizontal rule
}
%----------------------------------------------------------------------------------------
%	WHAT IS THE ASSIGNMENT NUMBER
%----------------------------------------------------------------------------------------
\author{Assignment 8  |   Math 710} 
\date{\normalsize\today} % Today's date or a custom date
\begin{document}
\maketitle % Print the title
%----------------------------------------------------------------------------------------
%	CHEETSHEET
%----------------------------------------------------------------------------------------
\begin{comment}
\begin{align*}
\int_E (f \cdot \chi A) &= \int_{E/A} (f \cdot \chi A) + \int_A (f \cdot \chi A) \text{ ,by linearity properties of the integral}.
\\  &= 0  + \int_A (f \cdot \chi A) \text{ , by definition of characteristic function}
\\  &=  \int_A (f \cdot \ 1) \text{ , by definition of characteristic function}
\end{align*} 
ok here is the cheetshee for piecewise functions
$$
f(n) =
\begin{cases}
n/2, & \text{if }n\text{ is even} \\
3n+1, & \text{if }n\text{ is odd}
\end{cases}
$$
\end{comment}	

%\begin{comment}
%----------------------------------------------------------------------------------------
%	PROBLEM 1
%----------------------------------------------------------------------------------------

\section*{Exercise 4.11}
\boldmath
\textbf{Show that a BV function is bounded.}
\unboldmath
\begin{proof}
WLOG, consider a function of BV over the interval $[a, b]$ with the partition defined by $P = \{a, x, b\}$.  We NTS that $|f(x)| \leq \infty$.  By Triangle Inequality, 
$$ |f(x)| = |f(x) - f(a) + f(a) | \leq |f(a)| + |f(x) - f(a)| + |f(b) - f(x)| = |f(a)| + V_{a}^{b}(P, f) \leq |f(a)| + V_{a}^{b}(f).$$
Since $V_{a}^{b}(f)$ is finite, $|f(a)| + V_{a}^{b}(f)$ is clearly finite as well.  Thus $f(x)$ is bounded.
\end{proof}

%----------------------------------------------------------------------------------------
%	PROBLEM 2
%----------------------------------------------------------------------------------------

\section*{Exercise 4.18}
\boldmath
\textbf{Let $\{I_1, \dots, I_n \}$ be a set of bounded intervals such that $m(I_1) \geq m(I_i)$ and $I_i \cap I_1 \neq 0$ for all $1 \leq i \leq n.$  Show that: \newline a)  $\bigcup_{i = 1}^{n} I_n$ is a bounded interval. 
\newline b)  $m(I_i) \geq \frac{1}{3}m\left( \bigcup_{i = 1}^{n} I_1 \right)$}
\unboldmath
\newline
\begin{proof}
a)     To show that the union is bounded, we need to show that inf $\bigcup_{i = 1}^{n} I_n$ and sup $\bigcup_{i = 1}^{n} I_n$ exist.
\newline
Since every interval is bounded, supremum exists for each $I_i$; it follows that sup $\bigcup_{i = 1}^{n} I_n$ is the supremum of this set of supremum.  For inf $\bigcup_{i = 1}^{n} I_n$, the process follows the same reasoning. 
\end{proof}
\begin{proof}
b)
Given the set of bounded intervals $\{I_1, \dots, I_n\}$ and the fact that $m(I_1) \geq m(I_i)$, we can select a nonempty subset $S$ such that $m(I_1) \geq m(I_i)$ for $i \in S$.  Then by Lemma 4.21, 
$m(I_1) \geq \frac{1}{3} m \left( \bigcup_{i = 1}^{n} I_i \right) $.
\end{proof}
%----------------------------------------------------------------------------------------
%	PROBLEM 3
%----------------------------------------------------------------------------------------

\section*{Exercise 4.19}
\boldmath
\textbf{Show that a bounded set $E$ that is not of measure zero has a positive outer measure, $m^*(E) >0$.}
\unboldmath
\begin{proof}
By hypothesis $E$ has measure; thus $m(E)$ must be positive by definition of measure.  By Definition 2.28, $m(E) = m^*(E) > 0$ and we are done.
\end{proof}

%----------------------------------------------------------------------------------------
%	PROBLEM 4
%----------------------------------------------------------------------------------------

\section*{Exercise 4.20}
\boldmath
\textbf{Let $A$ be a finite or countable subset of a bounded set $E$.  Show that $E$ is a set of measure zero if and only if $E \backslash A$ is a set of measure zero.}
\unboldmath
\begin{proof}
$(\Rightarrow)$
\newline
Let the bounded set $E$ be the union of an at most countable family of sets $\{E_i\}_{i \in J}$.  Since $A$ is a finite or countable subset of $E$, by Lemma 4.22, $m(E \textbackslash A) = 0$.
\newline
$(\Leftarrow)$
Let $m(E \textbackslash A)= 0$.  Recall $A$ is a finite or countable subset of $E$, so $m(A) =0$, and by Lemma 4.22, $m(E) = 0$.
\end{proof}

%\end{comment}


\end{document}