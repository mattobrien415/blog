%%%%%%%%%%%%%%%%%%%%%%%%%%%%%%%%%%%%%%%%%
% Short Sectioned Assignment
% LaTeX Template
% Version 1.0 (5/5/12)
%
% This template has been downloaded from:
% http://www.LaTeXTemplates.com
%
% Original author:
% Frits Wenneker (http://www.howtotex.com)
%
% License:
% CC BY-NC-SA 3.0 (http://creativecommons.org/licenses/by-nc-sa/3.0/)
%
%%%%%%%%%%%%%%%%%%%%%%%%%%%%%%%%%%%%%%%%%

%----------------------------------------------------------------------------------------
%	PACKAGES AND OTHER DOCUMENT CONFIGURATIONS
%----------------------------------------------------------------------------------------

\documentclass[paper=a4, fontsize=11pt]{scrartcl} % A4 paper and 11pt font size

\usepackage[T1]{fontenc} % Use 8-bit encoding that has 256 glyphs
\usepackage{fourier} % Use the Adobe Utopia font for the document - comment this line to return to the LaTeX default
\usepackage[english]{babel} % English language/hyphenation
\usepackage{amsmath,amsfonts,amsthm} % Math packages
\usepackage{verbatim} %for commenting out blocks

\usepackage{lipsum} % Used for inserting dummy 'Lorem ipsum' text into the template

\usepackage{sectsty} % Allows customizing section commands
\allsectionsfont{\centering \normalfont\scshape} % Make all sections centered, the default font and small caps

\usepackage{fancyhdr} % Custom headers and footers
\pagestyle{fancyplain} % Makes all pages in the document conform to the custom headers and footers
\fancyhead{} % No page header - if you want one, create it in the same way as the footers below
\fancyfoot[L]{} % Empty left footer
\fancyfoot[C]{} % Empty center footer
\fancyfoot[R]{\thepage} % Page numbering for right footer
\renewcommand{\headrulewidth}{0pt} % Remove header underlines
\renewcommand{\footrulewidth}{0pt} % Remove footer underlines
\setlength{\headheight}{13.6pt} % Customize the height of the header
%\numberwithin{equation}{section} % Number equations within sections (i.e. 1.1, 1.2, 2.1, 2.2 instead of 1, 2, 3, 4)
%\numberwithin{figure}{section} % Number figures within sections (i.e. 1.1, 1.2, 2.1, 2.2 instead of 1, 2, 3, 4)
%\numberwithin{table}{section} % Number tables within sections (i.e. 1.1, 1.2, 2.1, 2.2 instead of 1, 2, 3, 4)

\setlength\parindent{0pt} % Removes all indentation from paragraphs - comment this line for an assignment with lots of text
%----------------------------------------------------------------------------------------
%	TITLE SECTION
%----------------------------------------------------------------------------------------
\newcommand{\horrule}[1]{\rule{\linewidth}{#1}} % Create horizontal rule command with 1 argument of height
\title{	
\normalfont \normalsize 
%----------------------------------------------------------------------------------------
%	PUT YOUR NAME HERE 
%----------------------------------------------------------------------------------------
\textsc{Matt O'Brien} \\ [25pt] % Your university, school and/or department name(s)
\horrule{0.5pt} \\[0.4cm] % Thin top horizontal rule
%\huge  \\ % The assignment title
%\horrule{2pt} \\[0.5cm] % Thick bottom horizontal rule
}
%----------------------------------------------------------------------------------------
%	WHAT IS THE ASSIGNMENT NUMBER
%----------------------------------------------------------------------------------------
\author{Assignment V} 
\date{\normalsize\today} % Today's date or a custom date
\begin{document}
\maketitle % Print the title
%----------------------------------------------------------------------------------------
%	CHEETSHEET
%----------------------------------------------------------------------------------------
\begin{comment}
\begin{align*}
\int_E (f \cdot \chi A) &= \int_{E/A} (f \cdot \chi A) + \int_A (f \cdot \chi A) \text{ ,by linearity properties of the integral}.
\\  &= 0  + \int_A (f \cdot \chi A) \text{ , by definition of characteristic function}
\\  &=  \int_A (f \cdot \ 1) \text{ , by definition of characteristic function}
\end{align*} 
\end{comment}	
%----------------------------------------------------------------------------------------
%	PROBLEM 1
%----------------------------------------------------------------------------------------

\section*{Exercise 1}
\boldmath
\textbf{Show from first principles that if $f^1$, $f^2$,...,$f^k$ : $\mathbb{R^n} \rightarrow \mathbb{R}$ are convex (concave) functions with the same domain, and if $\omega_1$,...,$\omega_k$ are non-negative scalars, then the function $\omega_1 f^1 +$...$+ \omega_k f^k$ is also convex (concave). }
\unboldmath
\begin{proof}
Given $f^1$, $f^2$,...,$f^k$ as convex functions and $\omega_1$,...,$\omega_k$ as non-negative scalars, we need to show that $\omega_1 f^1 +$...$+ \omega_k f^k$ is also convex; that is, $$\sum_{i = 1}^k \omega_i f_i$$ is a convex function.
\newline
Using first principles, we consider a linear combination $\lambda x + (1 - \lambda)y$ where $\lambda \in [0, 1]$, $x, y \in $ the domain of all $f^k$.
\begin{align*}
\\f (\lambda x + (1 - \lambda)y &= \sum_{i=1}^k \omega_i f_i (\lambda x + (1 - \lambda )y) 
\\ &\leq \sum_{i=1}^k \omega_i (\lambda f_i (x) + (1- \lambda) f_i(y)) 
\\ &= \lambda \sum_{i =1}^k \omega_i f_i(x) = (1 - \lambda) \sum_{i =1}^k \omega f_i(y)
\\ &= \lambda f(x) + (1 - \lambda) f(x).
\end{align*}
Thus we have shown that convexity holds under this operation.
\end{proof}




%----------------------------------------------------------------------------------------
%	PROBLEM 2
%----------------------------------------------------------------------------------------

\section*{Exercise 2}
\boldmath
\textbf{}
\unboldmath
A)  Let $f(x) : \mathbb{R}^n \rightarrow \mathbb{R}$ be a convex function.  Show that for any $\alpha \in \mathbb{R}$ the set $\{ x \in \mathbb{R^n} :  f(x) \leq \alpha \}$ is a convex set.
\begin{proof}
Let $f(x)$ be convex, where $f(x): \mathbb{R^n} \rightarrow	\mathbb{R}.$  Let $\alpha \in \mathbb{R}.$  Let $\lambda \in [0, 1]$.  Consider the set $\{ x \in \mathbb{R^n}: f(x) \leq \alpha \} .$
\newline
Take two vectors from this set:  let \textbf{a}$ = (x_1, \dots , x_n)$, let \textbf{b}$ = (w_1, \dots, w_n) \in \{x \in \mathbb{R^n}: f(x) \leq \alpha \}, $ such that $f(x_1, \dots , x_n) < y,$ and $f(w_1, \dots , w_n) < v.$ 
\newline
We NTS that the convex combination of these two vectors is still within the set; we check directly by considering $\lambda(x_1, \dots , x_n) + (1 - \lambda)(w_1, \dots, w_n)$.  By distributing and grouping we get the vector $t = \lambda x_1 + (1 - \lambda)w_1 + \dots + x_n + (1 - \lambda)(w_1, \dots, w_n)$.
\newline
Recall from lecture the general form for convex functions:  $$f(\lambda (a) + (1- \lambda b) \leq \lambda f(a) = (1 - \lambda)f(b).$$  In our instance, and since $f(a) \leq y$ and $f(b) \leq v$, we write $$f(\lambda x_1 + (1 - \lambda ) w_1 + \dots \lambda x_n + (1 - \lambda)w_n) \leq \lambda f(x_1, \dots, x_n) + ( 1 - \lambda ) f (w_1, \dots, w_n). \leq \lambda y  + ( 1- \lambda ) v.$$
\end{proof}
\boldmath
\textbf{}
\unboldmath



B)  Show that the converse is not true.  Give an example of a non-convex function with the above sets for every $\alpha$ convex.
\begin{proof}
As a counterexample consider the non-concave function $f(x) = -e^x$, of which the epigraph is convex.
\end{proof}






%----------------------------------------------------------------------------------------
%	PROBLEM 3
%----------------------------------------------------------------------------------------

\section*{Exercise }
\boldmath
\textbf{The geometric mean function is $f(x_1, x_2, ..., x_n) = \left(\Pi_{i = 1}^{n} \right)^\frac{1}{n}$ defined on the positive orthant $\mathbb{R}_{>0}^n = (x_1, ..., x_n) : x_i > 0, i = 1, ..., n . $  Show that this function is concave by computing its Hessian and using similar methods as in the case of log-sum-exp function.}
\unboldmath
\begin{proof}
The Hessian of the geometric mean function is $\bigtriangledown^2 f(x) = \partial^2 f(x)$
\end{proof}




\begin{comment}
%----------------------------------------------------------------------------------------
%	PROBLEM 4
%----------------------------------------------------------------------------------------
\section*{Exercise   }
\boldmath
\textbf{}
\unboldmath
\begin{proof}
\end{proof}
%----------------------------------------------------------------------------------------
%	PROBLEM 5
%----------------------------------------------------------------------------------------
\section*{Exercise }
\boldmath
\textbf{}
\unboldmath
\begin{proof}
\end{proof}
\end{comment}
\end{document}