%%%%%%%%%%%%%%%%%%%%%%%%%%%%%%%%%%%%%%%%%
% Short Sectioned Assignment
% LaTeX Template
% Version 1.0 (5/5/12)
%
% This template has been downloaded from:
% http://www.LaTeXTemplates.com
%
% Original author:
% Frits Wenneker (http://www.howtotex.com)
%
% License:
% CC BY-NC-SA 3.0 (http://creativecommons.org/licenses/by-nc-sa/3.0/)
%
%%%%%%%%%%%%%%%%%%%%%%%%%%%%%%%%%%%%%%%%%

%----------------------------------------------------------------------------------------
%	PACKAGES AND OTHER DOCUMENT CONFIGURATIONS
%----------------------------------------------------------------------------------------

\documentclass[paper=a4, fontsize=11pt]{scrartcl} % A4 paper and 11pt font size

\usepackage[T1]{fontenc} % Use 8-bit encoding that has 256 glyphs
\usepackage{fourier} % Use the Adobe Utopia font for the document - comment this line to return to the LaTeX default
\usepackage[english]{babel} % English language/hyphenation
\usepackage{amsmath,amsfonts,amsthm} % Math packages
\usepackage{verbatim} %for commenting out blocks

\usepackage{lipsum} % Used for inserting dummy 'Lorem ipsum' text into the template

\usepackage{sectsty} % Allows customizing section commands
\allsectionsfont{\centering \normalfont\scshape} % Make all sections centered, the default font and small caps

\usepackage{fancyhdr} % Custom headers and footers
\pagestyle{fancyplain} % Makes all pages in the document conform to the custom headers and footers
\fancyhead{} % No page header - if you want one, create it in the same way as the footers below
\fancyfoot[L]{} % Empty left footer
\fancyfoot[C]{} % Empty center footer
\fancyfoot[R]{\thepage} % Page numbering for right footer
\renewcommand{\headrulewidth}{0pt} % Remove header underlines
\renewcommand{\footrulewidth}{0pt} % Remove footer underlines
\setlength{\headheight}{13.6pt} % Customize the height of the header
%\numberwithin{equation}{section} % Number equations within sections (i.e. 1.1, 1.2, 2.1, 2.2 instead of 1, 2, 3, 4)
%\numberwithin{figure}{section} % Number figures within sections (i.e. 1.1, 1.2, 2.1, 2.2 instead of 1, 2, 3, 4)
%\numberwithin{table}{section} % Number tables within sections (i.e. 1.1, 1.2, 2.1, 2.2 instead of 1, 2, 3, 4)

\setlength\parindent{0pt} % Removes all indentation from paragraphs - comment this line for an assignment with lots of text
%----------------------------------------------------------------------------------------
%	TITLE SECTION
%----------------------------------------------------------------------------------------
\newcommand{\horrule}[1]{\rule{\linewidth}{#1}} % Create horizontal rule command with 1 argument of height
\title{	
\normalfont \normalsize 
%----------------------------------------------------------------------------------------
%	PUT YOUR NAME HERE 
%----------------------------------------------------------------------------------------
\textsc{Matt O'Brien} \\ [25pt] % Your university, school and/or department name(s)
\horrule{0.5pt} \\[0.4cm] % Thin top horizontal rule
%\huge  \\ % The assignment title
%\horrule{2pt} \\[0.5cm] % Thick bottom horizontal rule
}
%----------------------------------------------------------------------------------------
%	WHAT IS THE ASSIGNMENT NUMBER
%----------------------------------------------------------------------------------------
\author{Assignment 7, Math 410} 
\date{\normalsize\today} % Today's date or a custom date
\begin{document}
\maketitle % Print the title
%----------------------------------------------------------------------------------------
%	CHEETSHEET
%----------------------------------------------------------------------------------------
\begin{comment}
\begin{align*}
\int_E (f \cdot \chi A) &= \int_{E/A} (f \cdot \chi A) + \int_A (f \cdot \chi A) \text{ ,by linearity properties of the integral}.
\\  &= 0  + \int_A (f \cdot \chi A) \text{ , by definition of characteristic function}
\\  &=  \int_A (f \cdot \ 1) \text{ , by definition of characteristic function}
\end{align*} 
ok here is the cheetshee for piecewise functions
$$
f(n) =
\begin{cases}
n/2, & \text{if }n\text{ is even} \\
3n+1, & \text{if }n\text{ is odd}
\end{cases}
$$
\end{comment}
%----------------------------------------------------------------------------------------
%	PROBLEM 11
%----------------------------------------------------------------------------------------

\section*{Exercise 4.3}
\boldmath
\textbf{Show that the function $$
f(x) =
\begin{cases}
x^2 sin \frac{1}{x^2}   & \text{if } x \neq 0 \\
0 & \text{if } x = 0
\end{cases}
$$
is not of a bounded variation over the interval $[-1, 1]$.}
\unboldmath
\begin{proof}
First, although the problem requests we consider $f(x)$ over the interval $[-1, 1]$, notice that $f(x) = f(-x)$.  Thus we need only consider $f(x)$ over $[0, 1]$.
\newline
To show that $f(x)$ is not a BV-function, we must construct a sequence of partitions, compute the variation over that sequence of partitions, and determine that the variatin is infinite.
\newline
We start by looking at the inner portion of the right side, $x^2$, and notice that this part of the composite function is bounded by $ \pm 1$.  Next we look at the inner  sin$\left( \frac{1}{x^2} \right)$ portion.  Notice that sin$\left(\frac{1}{x^2} \right)= 0$ when $x = \frac{1}{\sqrt{nx}}$ and sin$\left(\frac{1}{x^2} \right)= 1$ when $x = \frac{ \sqrt{\frac{2}{\pi}}}{ \sqrt{1 + 4j}}$, for all $j$ in $\mathbb{Z}$
\newline
We build the partition $P$ as follows:  let $P = \{0, 1\} \bigcup \{x = \frac{1}{\sqrt{j \pi}}, j \in \{1, \dots, k \}\} \bigcup \left\{ \frac{ \sqrt{\frac{2}{\pi}}}{ \sqrt{1 + 4j}}, j \in  {(0, 1, \dots, k) } \right\}$
\newline
We compute the variation: where $n =  $the number of elements in P, we have:
\begin{align*}
\\ V_0^1(P,f) &= \sum_{i = 1}^{n} \left| f(x_i) - f(x_{i -1} \right|
\\ &=  \sum_{j = j_0}^{k} x^2 
\\ &=  \sum_{j = j_0}^{k} \frac{ \frac{\sqrt{2}}{\pi}}{4j + 1}
\end{align*}
\newline
To determine that the variation is unbounded, we apply the Limit Comparison Test for harmonic series. Therefore, $V_{0}^{1}(P, f)$ diverges when $k \rightarrow \infty$, and we are done.
\end{proof}

%----------------------------------------------------------------------------------------
%	PROBLEM 2
%----------------------------------------------------------------------------------------
\section*{Exercise 4.4}
\boldmath
\textbf{Show that the function $$
f(x) =
\begin{cases}
x^2 cos \frac{1}{x}   & \text{if } x \neq 0 \\
0 & \text{if } x = 0
\end{cases}
$$
is a bounded variation over the interval $[0, 1]$.}
\unboldmath
\begin{proof}
We need to show that $f$ is a Lipshitz function.  First, note that $f'(x) = 2$cos$(\frac{1}{x}) + $sin$(\frac{1}{x})$ when $x \neq 0.$  Since $\left| \text{cos} (\frac{1}{x}) \right| $
and $ \left| \text{sin} (\frac{1}{x}) \right|$ are both bounded above by $1$ for all $x$, we know that $\left| f'(x) \right| = \left| 2x \text{cos} (\frac{1}{x}) + \text{sin}(\frac{1}{x}) \right| \leq \left|2x \text{cos}(\frac{1}{x})\right| + \left|\text{sin}(\frac{1}{x}) \right| \leq 2 \cdot 1 \cdot 1 + 1 = 3$.
\newline
Second, note that $f'(0) = \lim_{h \rightarrow 0} \frac{f(h) - f(0)}{h} = \lim_{h \rightarrow 0} \frac{{h^2}\text{cos}(\frac{1}{h})}{h} = \lim_{h \rightarrow 0} h \text{cos}(\frac{1}{h}).$ 
\newline
Since $|\text{cos}(\frac{1}{h})| \leq 1, |h|\cdot\text{cos}((\frac{1}{h}) = |h\text{cos}\frac({1}{h})| \leq |h| = 0$, by the Squeeze Theorem.  This shows that $f$ is Lipshitz and thus $f$ is of bounded variation.

\end{proof}

%----------------------------------------------------------------------------------------
%	PROBLEM 3
%----------------------------------------------------------------------------------------

\section*{Exercise 4.5}
\boldmath
\textbf{Let $f$ be a BV-function on $[a, b]$ and $[c, d]$ be a nontrivial subinterval of $[a, b]$.  Then $f$ is a BV-function over $[c, d]$.}
\unboldmath
\begin{proof}
Consider a partition $P$ of $[c, d]$ where $P = \{x_i = 1 \leq i \leq n \}.$  Assume that $f(x_i) - f(x_{i-1})$ is bounded with regard to $P$.  Then, the variation of $f$ over $[c, d]$, $V(f, [c, d]_)$ is the supremum of the set $S$ where $S = \sum_{i = 1}^{n} | \{f(x_i) - f(x_{i-1}) |$.  If this supremum exists and is finite, then we say that the function is a BV function over $[c, d].$
\end{proof}
%----------------------------------------------------------------------------------------
%	PROBLEM 4
%----------------------------------------------------------------------------------------
\section*{Exercise 4.6}
\boldmath
\textbf{Show that the sum, difference and product of two BV-functions is a BV-function.}
\unboldmath
\begin{proof}
Let $f$, $g$, be two BV functions over $[a, b]$.  Define a partition, $P = \{x_i = 1 \leq i \leq n \}.$  Then ,
\begin{align*}
\sum_{i = 1}^{n} | (f + g )(x_i) - (f + g)(x_{i-1}) | &= \sum_{i = 1}^{n} | \{ f(x_i) + g(x_i) \} - \{ f(x_{i - 1} + g(x_{i-1}) \} |
\\  &\leq \sum_{i = 1}^{n} | f(x_i) - f(x_{i - 1} | + \sum_{i = 1}^{n} | g(x_i) - g(x_{i - 1})|
\\  &\leq  V(f, P) + V(g, P)
\end{align*} 
$\Rightarrow V(f + g, P) \leq V(f, P) + V(g, P) \leq V(f, P) + V(g, P)
\Rightarrow V(f + g, P) \leq V(f, P) + V(g, P).$
Thus, $(f + g)$ is a function of bounded variation.
\newline To show that $(f - g)$ is of bounded variation, the proof is the same and $V(f - g) \leq V(f) + V(g)$
\newline
To show that the product of two functions of bounded variation is also of bounded variation, notice that both $f$ and $g$ are bounded, and thus there exists $K \in \mathbb{N}$ such that
$|f(x)| \leq k, |g(x) | \leq k, \forall x \in P$.
\newline
Thus, 
\begin{align*}
\sum_{i = 1}^{n} | (fg)(x_i) - (fg)(x_{i - 1}|&= \sum_{i = 1}^{n} | f(x_i)g(x_i) - f(x_{i -1}g(x{i - 1})|
\\ &= \sum_{i = 1}^{n} |f(x_i) \{g(x_i) - g(x_{i - 1} \} + g(x_{i-1} \{ f(x_i) - f(x_{i - 1}) \} |
\\ &\leq \sum_{i = 1}^{n} \{ | f(x_i) | |g(x_i)  - g(x_{i - 1} | + | g(x_{i - 1} | | f(x_i) - f(x_{i - 1}) | \}
\\ &\leq k \sum_{i = 1}^{n} |g(x_i) - g(x_{i-1}) | + k \sum_{i = 1}^{n} | f(x_i) - f(x_{i -1})|
\\ &\leq k V(g) + kV(f).
\end{align*}
the product of two functions of bounded variation is also of bounded variation.
\end{proof}

%----------------------------------------------------------------------------------------
%	PROBLEM 5
%----------------------------------------------------------------------------------------

\section*{Exercise 4.7}
\boldmath
\textbf{Let $f$ and $g$ be two BV-functions on $[a, b]$.  Show that $f/g$ is a BV-function provided that $g(x) \geq \alpha > 0$ on $[a, b]$.}
\unboldmath
\begin{proof}
Let $h = \frac{1}{g}$.  Then, for any partition $P$ over $[a, b]$, 
\begin{align*}
\sum \left| \frac{1}{f}(x_i) - \frac{1}{f}(x_{i - 1}) \right| &= \sum \left| \frac{1}{f(x_i)} - \frac{1}{f(x_{i-1}} \right|
\\   &= \sum \left | \frac{f(x_{i -1)} - f(x_i)}{f(x_i)f(x_{i -1})} \right |
\\	&\leq \frac{1}{k^2} \sum \left | f(x_{i-1} - f(x_i) \right |
\\	&\leq \frac{1}{k^2} V (f,P).
\end{align*}
This shows that $\frac{1}{g} = h$ is a BV-function.  Then, by the previous question, $f \cdot h = \frac{f}{g}$ is a BV-function.
\end{proof}



%----------------------------------------------------------------------------------------
%	PROBLEM 6
%----------------------------------------------------------------------------------------

\section*{Exercise 4.8}
\boldmath
\textbf{Show that the Dirichlet's function 
$$
f(n) =
\begin{cases}
1, & \text{if } x \text{ is a rational number} \\
0, & \text{otherwise}
\end{cases}
$$
on $[0, 1]$ is not of a bounded variation.
}
\unboldmath
\begin{proof}
We start by construction a partition $P = \left \{ a = x_0, x_1, \dots , x_{n + 2}\right \}$.
\newline
Let $x_1$ be an irrational number inbetween $a$ and $b$.  Then let the next number in the partition, $x_2$, be a rational number inbetween $x_1$ and $b$.  Since by Theorem 1.18 in the textbook by Wade from last semester, there is a rational number and an irrational number inbetween any two real numbers.  So we construct the partition in that manner until between we have an interval that alternates between rationals and irrationals.
\newline
Now we consider the sum $\sum_{i = 1}{n +2} \left | f(x_i) - f(x_{i -1}) \right |$.  This sum is bounded by the variation of $f$ over $[a, b]$.  Using this sum we state the following inequality:
\begin{align*}
\	V(f, P) &\geq \sum_{i=1}^{n+2} \left | f(x_i) - f(x_{i-1}) \right |
\\	&\geq \sum_{i=2}^{n+1} \left | f(x_i) - f(x_{i-1}) \right |
\\	&= \left | f(x_2) - f(x_1) \right | + \dots + \left | f(x_{n+1} - f(x_n) \right |
\\ &= 1 + 1 + 1 + \dots + 1
\\ &= n.
\end{align*}
Since $V(f, P) = n$, where we can choose an $n$ of any size.  Thus $V(f, P) = \infty$.
\end{proof}

%----------------------------------------------------------------------------------------
%	PROBLEM 7
%----------------------------------------------------------------------------------------

\section*{Exercise 4.9}
\boldmath
\textbf{Show that if $f$ has a bounded derivative on $[a, b]$, then $f$ is of bounded variation.}
\unboldmath
\begin{proof}
Since $f$ has a bounded derivative on $[a, b]$ we write $\left | f' \right | \leq M$ on $[a, b]$.  Now, consider any two points $x_{k-1}, x_k \in [a, b]$.  By the Mean Value Theorem, we have $ \left | f(x_k) - f(x_{k-1}) \right | \leq M \sum_{k=1}{n} \left | x_k - x_{k-1} \right |$.  Since this inequality is for any two arbitrary points, the for any partition we can write
$$ \sum_{k=1}^{n} \left | f(x_k) - f(x_{k-1}) \right | \leq M \sum_{k=1}^{n} \left | x_k - x_{k-1} \right | = M(b -a) \Rightarrow V_a^b f \leq M(b-a).$$
\end{proof}


\end{document}