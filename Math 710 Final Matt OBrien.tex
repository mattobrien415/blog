%%%%%%%%%%%%%%%%%%%%%%%%%%%%%%%%%%%%%%%%%
% Short Sectioned Assignment
% LaTeX Template
% Version 1.0 (5/5/12)
%
% This template has been downloaded from:
% http://www.LaTeXTemplates.com
%
% Original author:
% Frits Wenneker (http://www.howtotex.com)
%
% License:
% CC BY-NC-SA 3.0 (http://creativecommons.org/licenses/by-nc-sa/3.0/)
%
%%%%%%%%%%%%%%%%%%%%%%%%%%%%%%%%%%%%%%%%%

%----------------------------------------------------------------------------------------
%	PACKAGES AND OTHER DOCUMENT CONFIGURATIONS
%----------------------------------------------------------------------------------------

\documentclass[paper=a4, fontsize=11pt]{scrartcl} % A4 paper and 11pt font size

\usepackage[T1]{fontenc} % Use 8-bit encoding that has 256 glyphs
\usepackage{fourier} % Use the Adobe Utopia font for the document - comment this line to return to the LaTeX default
\usepackage[english]{babel} % English language/hyphenation
\usepackage{amsmath,amsfonts,amsthm} % Math packages
\usepackage{verbatim} %for commenting out blocks

\usepackage{lipsum} % Used for inserting dummy 'Lorem ipsum' text into the template

\usepackage{sectsty} % Allows customizing section commands
\allsectionsfont{\centering \normalfont\scshape} % Make all sections centered, the default font and small caps

\usepackage{fancyhdr} % Custom headers and footers
\pagestyle{fancyplain} % Makes all pages in the document conform to the custom headers and footers
\fancyhead{} % No page header - if you want one, create it in the same way as the footers below
\fancyfoot[L]{} % Empty left footer
\fancyfoot[C]{} % Empty center footer
\fancyfoot[R]{\thepage} % Page numbering for right footer
\renewcommand{\headrulewidth}{0pt} % Remove header underlines
\renewcommand{\footrulewidth}{0pt} % Remove footer underlines
\setlength{\headheight}{13.6pt} % Customize the height of the header
%\numberwithin{equation}{section} % Number equations within sections (i.e. 1.1, 1.2, 2.1, 2.2 instead of 1, 2, 3, 4)
%\numberwithin{figure}{section} % Number figures within sections (i.e. 1.1, 1.2, 2.1, 2.2 instead of 1, 2, 3, 4)
%\numberwithin{table}{section} % Number tables within sections (i.e. 1.1, 1.2, 2.1, 2.2 instead of 1, 2, 3, 4)

\setlength\parindent{0pt} % Removes all indentation from paragraphs - comment this line for an assignment with lots of text
%----------------------------------------------------------------------------------------
%	TITLE SECTION
%----------------------------------------------------------------------------------------
\newcommand{\horrule}[1]{\rule{\linewidth}{#1}} % Create horizontal rule command with 1 argument of height
\title{	
\normalfont \normalsize 
%----------------------------------------------------------------------------------------
%	PUT YOUR NAME HERE 
%----------------------------------------------------------------------------------------
\textsc{Matt O'Brien} \\ [25pt] % Your university, school and/or department name(s)
\horrule{0.5pt} \\[0.4cm] % Thin top horizontal rule
%\huge  \\ % The assignment title
%\horrule{2pt} \\[0.5cm] % Thick bottom horizontal rule
}
%----------------------------------------------------------------------------------------
%	WHAT IS THE ASSIGNMENT NUMBER
%----------------------------------------------------------------------------------------
\author{Final MATH 710} 
\date{\normalsize\today} % Today's date or a custom date
\begin{document}
\maketitle % Print the title
%----------------------------------------------------------------------------------------
%	CHEETSHEET
%----------------------------------------------------------------------------------------
\begin{comment}
\begin{align*}
\int_E (f \cdot \chi A) &= \int_{E/A} (f \cdot \chi A) + \int_A (f \cdot \chi A) \text{ ,by linearity properties of the integral}.
\\  &= 0  + \int_A (f \cdot \chi A) \text{ , by definition of characteristic function}
\\  &=  \int_A (f \cdot \ 1) \text{ , by definition of characteristic function}
\end{align*} 
ok here is the cheetshee for piecewise functions
$$
f(n) =
\begin{cases}
n/2, & \text{if }n\text{ is even} \\
3n+1, & \text{if }n\text{ is odd}
\end{cases}
$$
\end{comment}	

%\begin{comment}
%----------------------------------------------------------------------------------------
%	PROBLEM 1
%----------------------------------------------------------------------------------------
\section*{Exercise 1}
\boldmath
\textbf{Let $f$ be a function of bounded variation on $[a, b]$ and $p \geq
1.$  Show that the function $|f|^{p}$ is also of a bounded variation
on $[a, b]$.}
\unboldmath
\begin{proof}
Since $f$ is a BV function, $\exists M > 0$ such that $|f| < M$. We proceed as normal by constructing a partition, $P = \{a = x_0, \dots x_n = b \}$.  Select a integer, $p'$ which satisfies the inequality $1 \leq p \leq p$.  Now consider the variation over $[a, b]$ of $f$:
\begin{align*}
\ V_{a}^{b}(P, |f|^p)&= \sum_{i = 1}^{n} \left| |f(x_i)|^p - |f(x_{i -1})|^p \right| 
\\ &=  \sum_{i =0}^{n-1} \left( (| |f(x_{i})|- |f(x_{i-1})||)(||f(x_i)|^{p-1} + |f(x_i)|^{p-2}|f(x_{i-1})| + \dots + |f(x_i)||f(x_{i-1})^{p-2} + |f(x_{i-1})|^{p-1}|)\right)
\\ &= \sum_{i=1}^{n} \left((||f(x_i)| - |f(x_{i-1})||) \sum_{j=1}^{p'}|f(x_i)|^{p-j}|f(x_{i-1})|^{j-1} \right)
\\ &\leq \sum_{i=1}^{n} \left( |f(x_i) - f(x_{i-1}) | \sum_{j=1}^{p'}|f(x_i)|^{p-j}|f(x_{i-1})|^{j-1} \right) \text{    by reverse triangle inequality,}
\\ &\leq \sum_{i=1}^{n} \left( |f(x_i) - f(x_{i -1})| \sum_{j=1}^{p'}M^{p-j}M^{j-1} \right) \text{  since } |f(x)| \text{ bounded by } M
\\ &= \sum_{i=1}^{n} \left( |f(x) - f(x_{i-1}) | \sum_{j=1}^{p'}M^{p-1} \right)
\\ &= \sum_{i=1}^{n} \left( |f(x_i) - f(x_{i-1})| \cdot p'M^{p-1} \right)
\\ &= p'M^{p-1} \sum_{i=1}^{n} |f(x_i) - f(x_{i-1})|
\\ &= p'M^{p-1}V_{a}^{b}(P, f)< Cp'M^{p-1} \text{ by defintion of Var(f) and definition of BV function.}
\end{align*}
This holds for some constant $C$.  Note that $Cp'M^{p-1}$ is the product of bounded elements and since $P$ is any arbitrary partition,  $Cp'M^{p-1}$ is therefore bounded about by it's supremum.  Therefore, $|f|^p$ is a function of bounded variation.
\end{proof}

%----------------------------------------------------------------------------------------
%	PROBLEM 2
%----------------------------------------------------------------------------------------
\section*{Exercise 2}
\boldmath
\textbf{Let $f$ be a continuous function on $[a, b]$ such that $|f|$ is a
BV-function on $[a, b]$.  Prove that $f$ is also a BV function on $[a,
b]$.  Show that continuity of $f$ is an essential assumption.}
\unboldmath
\begin{proof}
Consider a particular partition, $P' = \{a = x_0, x_1, \dots , x_n = b \}$ like usual.  The goal is to set these points so that we partition the domain in such a way where
the function is separated into intervals that are either positive or negative.
To achieve this, recall the Intermediate Value Theorem:
\newline
For continuous $f \in [a, b]$, $\exists$ a number $c \in (a, b)$ for which $f(c) = k$ where $k$ is between the function evaluated at the endpoints.  
\newline
This means that we can repeatedly apply this theorem over every interval defined by the partition $P'$, creating as many refinements as necessary so that $f(x) \geq 0 $ for all $x \in [x_i, x_{i+1}]$, or $f(x) \leq 0$ for all $x \in [x_i, x_{i+1}]$.
\newline
Therefore, for each of these intervals constructed in $P'$, 
$$||f(x_{i+1})| - |f(x_i)|| = |f(x_{i+1}) - f(x_i)|.$$
Further, for any other arbitrary partition $P$, we need only be concerned with $P \cap P'$, since $V_{a}^{b}(f, P) \leq V_{a}^{b}(f, P' \cap P).$
\newline
Since $|f|$ is a function of bounded variation, $V_{a}^{b}(|f|) \geq \sum ||f(x_i+1)| - |f(x_i)|| = \sum|f(x_{i+1} - f(x_i)|.$  So, finally,
$$V_{a}^{b}(f, P) \leq V_{a}^{b}(f, P' \cap P) \leq V_{a}^{b}(|f|, P' \cap P) \Rightarrow V_{a}^{b} \leq V_{a}^{b}(|f|).$$
\newline
To show that continuity is an essential part of determining if a function is of bounded variation, we define a function over $[0, 1]$, $f$ for which $f$ is not absolutely continuous but for $|f|$, $V_{0}^{1}(|f|)$ is finite.
\newline
$$
f(x) =
\begin{cases}
\frac{1}{2}, & \text{if }x \text{ is rational} \\
-\frac{1}{2}, & \text{if }x\text{ is irrational}
\end{cases}
$$
So, $|f(x)| = \frac{1}{2}$, which is constant and has bounded measure, whereas $f(x)$ fails the conditions for BV functions given by Theorem $4.17$, which states that a BV function is continuous everywhere except for a finite or countable number of points in it's closed domain.
\end{proof}
%----------------------------------------------------------------------------------------
%	PROBLEM 3
%----------------------------------------------------------------------------------------

\section*{Exercise  3 }
\boldmath
\textbf{Show that if $f$ and $g$ are BV-functions on $[a, b]$, then so is
$h(x) = max\{f(x), g(x)\}$ on the same interval.}
\unboldmath
Since $f(x)$ is BV, then by definition,
\newline
$V_{a}^{b}(f) = \sum_{k=1}^{n}|f(x_k) - f(x_{k-1})| = C_1$, and
\newline
$V_{a}^{b}(g) = \sum_{k=1}^{n}|g(x_k) - g(x_{k-1})| = C_2.$
To show that the function $h(x) = max\{f(x), g(x)\}$ is of bounded variation,
To show that the function $h(x) = max\{f(x), g(x)\}$ is of bounded variation,
$$|max\{f(x_i), g(x_i)\} - max\{f(x_{i-1}), g(x_{i-1})\}| \leq
|f(x_{i-1}, x_i)| + |g(x_{i-1}, x_i)|$$
Therefore,
$$\sum_{k=1}^{n}|max\{f(x_i), g(x_i)\} - max\{f(x_{i-1}),
g(x_{i-1})\}| \leq \sum_{k=1}^{n}|f(x_{i-1}, x_i)| + |g(x_{i-1}, x_i)|
\Rightarrow V_{a}^{b}(h) \leq C_1 + C_2.$$
Thus $V_{a}^{b}(h) \leq C_1 + C_2$

%----------------------------------------------------------------------------------------
%	PROBLEM 4
%----------------------------------------------------------------------------------------

\section*{Extra Credit Homework Exercise from lecture, Dec 6 }
\boldmath
\textbf{Show that the function $$
f(x) =
\begin{cases}
xcos\frac{\pi}{x}& \text{if }x \in (0,1] \\
0, & \text{if } x =0
\end{cases}
$$ is not a BV function.  Hints from class:  use technique from proof of $4.37$, and consider harmonic series.}
\unboldmath
\begin{proof}
To show this function is not BV, we construct a sequence of partitions as follows:
\newline
for each $m \in \mathbb{N}$ define $P_m = \{0, \frac{1}{2m}, \frac{1}{2m-1}, \dots, \frac{1}{3}, \frac{1}{2}, 1 \}$.  To check the function we evaluate it at those partition points and find $f(P_m) = \{0, \frac{1}{2m}, \frac{1}{2m-1}, \dots, -\frac{1}{3}, -\frac{1}{2}, -1 \}$ 
\newline
Next we consider the variation of $f$:
\begin{align*}
\sum_{i=1}^{n} \left| f(x_i) - f(x_{i-1} \right| &= \left| \frac{1}{2m} - 0) \right| + \left|-\frac{1}{2m-1} - \frac{1}{2m} \right| + \left|\frac{1}{2m-2} \right| + \dots + \left| -\frac{1}{3} - \frac{1}{4} \right| + \left| \frac{1}{2} + \frac{1}{3} \right| + \left| -1 -\frac{1}{2} \right|
\\ &= \frac{1}{2m} + \frac{1}{2m-1} + \frac{1}{2m} + \frac{1}{2m} + \frac{1}{2m-2m} + \frac{1}{2m-1} + \dots + \frac{1}{3} + \frac{1}{4} + \frac{1}{2} + \frac{1}{3} + 1 + \frac{1}{2}
\\ &= 2(\frac{1}{2m} + \frac{1}{2m-1} + \dots +  \frac{1}{2}) + 1
\end{align*}
This is a harmonic series in the form of $\sum_{n=2}^{\infty} \frac{1}{n}$ which is divergent.  So no matter what $M$ we choose for a bound, we can construct a partition for which the variation is unbounded.
\end{proof}
\end{document}