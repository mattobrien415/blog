%%%%%%%%%%%%%%%%%%%%%%%%%%%%%%%%%%%%%%%%%
% Short Sectioned Assignment
% LaTeX Template
% Version 1.0 (5/5/12)
%
% This template has been downloaded from:
% http://www.LaTeXTemplates.com
%
% Original author:
% Frits Wenneker (http://www.howtotex.com)
%
% License:
% CC BY-NC-SA 3.0 (http://creativecommons.org/licenses/by-nc-sa/3.0/)
%
%%%%%%%%%%%%%%%%%%%%%%%%%%%%%%%%%%%%%%%%%

%----------------------------------------------------------------------------------------
%	PACKAGES AND OTHER DOCUMENT CONFIGURATIONS
%----------------------------------------------------------------------------------------

\documentclass[paper=a4, fontsize=11pt]{scrartcl} % A4 paper and 11pt font size

\usepackage[T1]{fontenc} % Use 8-bit encoding that has 256 glyphs
\usepackage{fourier} % Use the Adobe Utopia font for the document - comment this line to return to the LaTeX default
\usepackage[english]{babel} % English language/hyphenation
\usepackage{amsmath,amsfonts,amsthm} % Math packages
\usepackage{verbatim} %for commenting out blocks

\usepackage{lipsum} % Used for inserting dummy 'Lorem ipsum' text into the template

\usepackage{sectsty} % Allows customizing section commands
\allsectionsfont{\centering \normalfont\scshape} % Make all sections centered, the default font and small caps

\usepackage{fancyhdr} % Custom headers and footers
\pagestyle{fancyplain} % Makes all pages in the document conform to the custom headers and footers
\fancyhead{} % No page header - if you want one, create it in the same way as the footers below
\fancyfoot[L]{} % Empty left footer
\fancyfoot[C]{} % Empty center footer
\fancyfoot[R]{\thepage} % Page numbering for right footer
\renewcommand{\headrulewidth}{0pt} % Remove header underlines
\renewcommand{\footrulewidth}{0pt} % Remove footer underlines
\setlength{\headheight}{13.6pt} % Customize the height of the header

%\numberwithin{equation}{section} % Number equations within sections (i.e. 1.1, 1.2, 2.1, 2.2 instead of 1, 2, 3, 4)
%\numberwithin{figure}{section} % Number figures within sections (i.e. 1.1, 1.2, 2.1, 2.2 instead of 1, 2, 3, 4)
%\numberwithin{table}{section} % Number tables within sections (i.e. 1.1, 1.2, 2.1, 2.2 instead of 1, 2, 3, 4)

\setlength\parindent{0pt} % Removes all indentation from paragraphs - comment this line for an assignment with lots of text

%----------------------------------------------------------------------------------------
%	TITLE SECTION
%----------------------------------------------------------------------------------------

\newcommand{\horrule}[1]{\rule{\linewidth}{#1}} % Create horizontal rule command with 1 argument of height

\title{	
\normalfont \normalsize 
\textsc{Matt O'Brien} \\ [25pt] % Your university, school and/or department name(s)
\horrule{0.5pt} \\[0.4cm] % Thin top horizontal rule
%\huge  \\ % The assignment title
%\horrule{2pt} \\[0.5cm] % Thick bottom horizontal rule
}

\author{Midterm 1 corrections Math 710} 


\date{\normalsize\today} % Today's date or a custom date

\begin{document}

\maketitle % Print the title


%----------------------------------------------------------------------------------------
%	PROBLEM 1
%----------------------------------------------------------------------------------------

\section*{Exercise 1}
\begin{comment}
\textbf{Let X be a closed subset of [0,1] such that m(X) = 1.  Prove that X = [0, 1].}
\newline
\newline
Let $X$ be closed subset of $[0,1]$ such that $m(X) = 1$.  We need to show that $X= [0,1]$.  Let $a=inf(X)$ and let $b=sup(X)$.  Suppose, by contradiction, that $X \neq [0, 1]$.  Then, there a 3 cases for which $X=[0, 1]$.
\newline
Case 1:  $a > 0$.  By definition of measure, $m(X) = (b -a) - m([a, b]\backslash X) = 1$, but $(b - a) < 1$ and $m(X)$ cannot equal $1$.
\newline
Case 2:  $b > 1$.  Again, be definition of measure, we fnd that $(b - a) < 1$ and $m(X)$ cannot equal $1$.
\newline
Case 3:  $a = 0, and b = 0$, since $b - a = 1$, and since $X = [0, 1]$, then $[0, 1]\textbackslash X = 0$ which implies that $X$ must be equal to $[0, 1]$.
\end{comment}

%----------------------------------------------------------------------------------------
%	PROBLEM 2
%----------------------------------------------------------------------------------------

\section*{Exercise 2}
\textbf{Show that a set E is measurable if and only if for each epsilon >0, there is a closed set F and an open set G for which F is a subset of E and E is a subset of G, and the outer measure of G - F is less than epsilon.}
\newline
$\Rightarrow$
\newline
Let $\epsilon < \frac{2}{\epsilon`}$.
Let $E$ be measurable.  Then, $E$ is bounded, and by exercise 2.13, and so we can find an open set $G \supseteq E$ such that $m(G) < m_*(E) + \epsilon`$.
\newline
By exercise 2.14, $\exists F \subseteq E$ such that $m(F) < m(E) + \epsilon`$.  Then, $m(G) - m(F) < m^*(E) - m_*(E) + 2\epsilon`$.  Since $m^*(E)-m_*(E) =0$, $m(G) - m(F) < 2\epsilon` < \epsilon$.  By theorem 2.31, $m(G \textbackslash F) < \epsilon$, and so by definition of measure, $m^*(G\textbackslash F)< \epsilon$.
\newline
$\Leftarrow$
\newline
We are given that $F$ is a closed set and $G$ is an open set.  $m^*(G\textbackslash F) < \epsilon$, and $F \subseteq E \subseteq G$.
\newline
First, note that outer measure exists, which allows us to surmise that $G$, $F$ are bounded.
\newline
We know that $m^*(G \textbackslash F) < \epsilon$ by definition of measure.
\newline
So, $G \textbackslash E \subseteq G \textbackslash E$, and $E \textbackslash F \subseteq G \textbackslash F$.
\newline
By theorem 2.24, $m^*(G \textbackslash E) \leq m^*(G \textbackslash F)$, and $m^*(E \textbackslash F)\leq m^*(G \textbackslash F)$.
\newline
Since $m^*(G \textbackslash F) < \epsilon$, then $m^*(G \textbackslash E) < \epsilon$.
By the results of exercise 2.23, and since $E$ is bounded, $m^*(G \textbackslash E) < \epsilon$ and $m^*(E \textbackslash F) \Rightarrow G$ is measurable. 



%----------------------------------------------------------------------------------------
%	PROBLEM 3
%----------------------------------------------------------------------------------------

\section*{Exercise 3  }
\textbf{Prove that a function f on [a,b] is measurable iff f inverse(U) is measurable for any open set U of R.}
\newline
\newline
$\Rightarrow$
\newline
By Theorem 2.4i, and Theorem 2.4iii, we have 2 open sets:
\newline
$ J = \{ x:  f(x) <j \}$
\newline
$K = \{ x:  f(x) > k \} ,$ where $j<k$.
\newline
Consider $J \cap K$ which is also open.
\newline
Let $U = J \bigcap K = \bigcup_{i=1}^{n}. (j_i, k_i)$, so that $f^{-1} \left( \bigcup \{ x \in [a, b] : f(x) \in U  \} \right) = f^{-}(U)$. 
\newline
\newline
$\Leftarrow$
\newline
Let $f^{-1}(U)$ be measurable.
\newline
Let $U = \left( c, \infty \right)$.
\newline
Then, $f^{-1}(U)$ measurable $\rightarrow \left\{x: f(x) > c \right\}$ is measurable.
\newline
By Theorem 2.48, $f$ is measurable.
\end{document}